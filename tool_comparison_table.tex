\documentclass{article}
\usepackage{amssymb,booktabs,graphicx,hyperref,pifont}
\usepackage[dvipsnames]{xcolor} 
\newcommand{\cmark}{{\color{Green} \ding{51}}}
\newcommand{\xmark}{{\color{Red} \ding{55}}}
\newcommand{\mcrot}[4]{\multicolumn{#1}{#2}{\rlap{\rotatebox{#3}{#4}~}}}
\newcommand{\rotangle}{45}
%
\begin{document}
\begin{table}[]
\begin{tabular}{*{12}l}
\multicolumn{1}{l}{ } & \mcrot{1}{l}{\rotangle}{Arbitrary distributed load functions} & \mcrot{1}{l}{\rotangle}{Full GUI (no code required)} & \mcrot{1}{l}{\rotangle}{Arbitrary number of loads} & \mcrot{1}{l}{\rotangle}{Free (as in free beer)} & \mcrot{1}{l}{\rotangle}{Open Source} & \mcrot{1}{l}{\rotangle}{Featured theory module} & \mcrot{1}{l}{\rotangle}{Detailed solution procedure} & \mcrot{1}{l}{\rotangle}{Programmable interface} & \mcrot{1}{l}{\rotangle}{Spring Supports} & \mcrot{1}{l}{\rotangle}{Any DOF combination for Supports} & \mcrot{1}{l}{\rotangle}{Any number of supports} \\ %\hline
\cmidrule{2-12}
{IndeterminateBeam}  	& \cmark & \cmark & \cmark & \cmark & \cmark & \cmark & \xmark & \cmark & \cmark & \cmark & \cmark  \\%\hline
{BeamBending} 		& \cmark & \xmark & \cmark & \cmark & \cmark & \cmark & \xmark & \cmark & \xmark & \xmark & \xmark \\ %\hline
{Beam Calculator Online}				& \xmark & \cmark & \cmark & \cmark & \xmark & \xmark & \xmark & \xmark & \xmark & \xmark & \xmark \\ %\hline
{Beam Guru} 							& \xmark & \cmark & \cmark & \xmark & \xmark & \xmark & \cmark & \xmark  & \xmark & \xmark & \cmark  \\ %\hline
{MechaniCalc}			& \xmark & \cmark & \cmark & \xmark & \xmark & \cmark & \xmark & \xmark  & \xmark & \cmark & \cmark \\ %\hline
{SkyCiv Beam} & \xmark & \cmark & \cmark & \xmark & \xmark & \xmark & \cmark & \xmark  & \xmark & \xmark & \xmark  \\ %\hline
{Steel Beam Calculator} 			& \xmark & \cmark & \cmark & \xmark & \xmark & \xmark & \xmark & \xmark  & \xmark & \xmark & \xmark  \\ %\hline
{Structural Beam Calculator}& \xmark & \cmark & \xmark & \cmark & \xmark & \xmark & \xmark & \xmark  & \xmark & \xmark & \xmark  \\ %\hline
{WebStructural} 	& \xmark & \cmark & \cmark & \xmark & \xmark & \xmark & \xmark & \xmark  & \xmark & \xmark & \cmark  \\ %\hline
{ClearCalcs} 			& \xmark & \cmark & \cmark & \xmark & \xmark & \xmark & \xmark & \xmark  & \cmark & \xmark & \cmark  \\ %\hline
\bottomrule
\end{tabular}
\caption{Summary of feature comparison with existing packages}
\end{table}
\end{document}
